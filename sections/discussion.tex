\section{Discussion}
\label{sec:discussion}

In this paper, we described re-using \Afrac{} as a reliable control variable for detachment control provided that real-time ion saturation current measurements are available from the Langmuir probes and the strike point is controlled well enough that such an array can be used to estimate peak $I_{sat}$.
It can be seen from panels (a) and (c) in Fig.\ref{fig:sysid_afrac}, Fig.\ref{fig:sysid_sm}, and Fig.\ref{fig:detctrl_sm} that when the total injected impurity amount crosses a rough threshold of about $1\times10^{20}$ particles, the plasma boundary shape is deformed such that strike point on outer divertor starts drifting downwards (and the inner strike point moves upwards) even though the X-point is held in place by the magnetic shape control system.
Thus, the \Afrac{} controller is best suited with an outer strike point control system commissioned on the device, as was demonstrated in the previous carbon divertor \cite{Eldon_2022_PPCF}.
Unfortunately, direct strike point position control (rather than X-point position control), was not yet commissioned for the new divertor at the time of these tests.
It is also important to keep note of the position of strike point and the width of the ion saturation current profile on the divertor.
In our case, we estimated that the ion saturation current profile width is about 16~mm, just enough to ensure that at least one of the Langmuir probes is always inside the wetted area from the \ac{SOL} plasma.
Even then, it can be seen that at around 12.4~s in Fig.\ref{fig:sysid_afrac}a and at around 10~s in Fig.\ref{fig:detctrl_afrac}a that as the strike point moves from OD9 ($Z=-1.225$~m) to OD11($Z=-1.2125$~m), the corresponding \Afrac{} value shows a sharp decline.
Sudden jumps in divertor state at the entry to detachment have been observed as real phenomena \cite{mclean_2015_jnm,Eldon_2017_NF}, but the simultaneous strike point motion could have added an unknown artifact to any real trend in the divertor behavior due to the peak passing between two probes at about the same time.
Fortunately, the excursion in strike point position is brief and the sudden drop in \Afrac{} is limited in magnitude, and the controller continued functioning.
This is something to watch out for as the sudden change is a departure from the simple system identification and sudden changes have the potential to excite unstable oscillations of the \ac{UGF}.
For future applications of this controller, we are working on including real-time spatial analysis of the $J_{sat}$ profile, potentially informed with profile shapes from high-fidelity simulations from SOLPS-ITER or UEDGE.

In our experimental session, since strike point control had not been commissioned, we attempted real-time empirical profile analysis with strike point sweeps.
But we found that the actuation strength and response time of the poloidal field coils at KSTAR do not allow for large enough and fast enough strike point sweeps.
Slower strike point sweeps are of course possible, but these will not support control and may as well be treated with post-shot analysis.
This result, that real-time strike point sweeps will have trouble supporting real-time divertor profile analysis for control purposes, is expected to hold for all tokamaks using low-temperature superconducting coils, as these coils tend to respond more slowly than copper, tend to be fewer in number, and have a larger minimum distance from the plasma to accommodate their insulation.
From our preparatory studies, real-time $J_{sat}$ profile analysis for detachment control does appear to be viable for machines with copper poloidal field coils like \mbox{DIII-D}, although the faster coil responses typical of these machines can facilitate tighter strike point position control and thus the need for real-time profile analysis for probe-based detachment control is minimal.
We have not studied how the shape control response times of machines with high temperature superconducting coils might factor into these considerations.

It should be noted that in the application of \Afrac{} controller method on KSTAR, tuning the overall factor to \Afrac{} so that it reports 1.0 when fully attached was trickier than the case for full carbon environment KSTAR \cite{Eldon_2022_PPCF}.
We noticed offsets in the outputs of Langmuir probes which changed from shot to shot, and thus ensuring the correct normalizing factor for \Afrac{} became harder.
The discharges in the tungsten divertor also took longer to achieve their final shape than the setup we used in the carbon divertor, producing a larger quantity of meaningless data that may have added to confusion.
Furthermore, divertor shape control was not yet commissioned to the extremely high quality that was achieved at the end of operations in carbon divertor, so distractions due to strike point motions were also present.
In retrospect, we find that we can unify all the discharges with one correction factor for for \Afrac{} and \Afrac{} values have been renormalized for plots that show multi-discharge comparisons.
After this experience, we have now added an online offset estimator and subtraction for all probe signals, which measures the offset before the plasma breakdown and ensures that the zero offset is correct on the probes.
This issue is likely due to electrical connectivity problems with the probes which also showed other issues during the campaign, but still, this experience should be noted for future reproduction and improvements.


Another point of uncertainty in \Afrac{} model could be the magnetic connection length between the upstream (outboard midplane) and divertor in the 2PM \cite{Leonard_2018_PPCF}.
This length is kept fixed in the model and while we did not change the plasma boundary shape much from the previous test \cite{Eldon_2022_PPCF} in carbon divertor KSTAR, it still makes it a potential source of error in wider usage in future.
The real-time equilibrium calculations during shot provided by RTEFIT is being upgraded to also output this magnetic connection length so that in future the model gets more accurate and time varying information about this important parameter.

Although Langmuir probes might not be able to survive future burning plasma experiments, they are still a valuable tool for studying detachment control experiments for ease of installation and operation in experimental devices.
Even in burning plasma devices, sacrificial Langmuir probes can be used in initial device commissioning and preparation of base scenarios at low power, although they are expected to fail shortly after exposure to the intensity of full heat and neutron fluxes.
Knowing what a stable detached scenario should be like could significantly decrease the time required to commission controllers based on other control variables, and give a baseline level of detachment control performance to compare them to.
Thus, Langmuir probe based control might provide a foothold in future device commissioning, as it has been shown to be useful on many devices \cite{Eldon_2021_NME, Guillemaut_2017_PPCF, Yuan_2020_FED, Khodunov_2021_PPCF}.
The good results from the \Afrac{} controller as seen in Fig.\ref{fig:detctrl_afrac} could also motivate further research in similar biased electrode measurement methods of SOL plasma such as biased divertor plates \cite{Toi_2023_NF, Cui_2024_NF} which behave like larger area Langmuir probes and can withstand harsher conditions in comparison to small tip area probes.

We also demonstrated using a machine-learning-based surrogate model, DivControlNN, which infers from a large database of 2D UEDGE simulations for estimating hard to infer quantities in the plasma, such as heat flux on the divertor, for controlling detachment level with real-time feedback.
As of the writing of this manuscript, this detachment control method is the first of its kind ever implemented and will act as a stepping stone for future deployments.
This is an important step in the direction of achieving detachment control in future burning plasma reactors which would have very limited means of measuring the detachment level due to space constraints and harsh environment.

We have identified critical weak points in the prototype of DivControlNN and the control infrastructure required to utilize this model, and we are working on improving these aspects for future tests.
One likely mistake we made during the experiments was treating the long dead time reported by DivControlNN heat flux output in response to gas puff (Fig.\ref{fig:sysid_sm}) as an overestimate.
Since DivControlNN had a constant zero impurity fraction concentration, it was solely relying on line integrated core electron density information for responding to changes.
While ion saturation currents provide local divertor information fast, the core electron density response to gas puff would have additional transport timescales and thus DivControlNN output might truly have a larger daed time.
This could have resulted in the high controller gain that saturated the gas response in the test (Fig.\ref{fig:detctrl_sm}).

We are in the process of creating a new 2D UEDGE database of KSTAR with a tungsten divertor and considering multiple charged states of additional impurities such as nitrogen, neon, and argon.
New models would be trained on the expanded database and acquired experimental data from this campaign, with the input of injected gas flow instead of impurity fraction to simplify the use case of these models.
We would also work with the KSTAR team to improve PCS communication infrastructure so that accurate real-time values of injected power are available to our models.
Accuracy would be further improved by removing core radiated power $P_{rad,core}$ from the power that is input from the core plasma to the UEDGE computation region.
Omitting $P_{rad,core}$ is a more serious error when significant tungsten accumulates near the magnetic axis and radiates strongly there (as was seen in \ref{fig:prad_2d}c) and this should be included.

The initial success of the neural network surrogate model in detachment control motivates and corroborates similar studies, simulations, and training of other models for providing fast estimates of plasma parameters, for quick decision-making in the control room during experiments, as well as, for potential use in other control systems where important plasma properties are often not accessible directly.
A neural network based control system approach has already been demonstrated in magnetic shape control \cite{Degrave_2022_Nature}.
For \ac{SOL} plasma predictions, machine learning surrogate models were first pioneered using 1D UEDGE simulations \cite{Zhu_2022_JPP}, serving as the proof-of-principle study that paved the way for DivControlNN presented here, which is based on 2D UEDGE simulations.
More recently, model based on Hermes-3 \cite{Dudson_2024_CPC} simulations of MAST-U \cite{Holt_2024_NF} and neural partial differential equation solver for TCV \cite{Poels_2023_NF} have been reported and are under further development.

Another major focus of future experiments would be to use noble gases in detachment control.
N$_2$, while being excellent at cooling the SOL plasma in conventional tokamaks, would not be a good impurity to seed in tritium fueled plasma due to the formation of tritiated ammonia \cite{Pitts_2019_NME}.
Such tritiated ammonia would require additional tritium reclaiming processes which would reduce the duty cycle of reactors, as well as, pose additional risks in handling a radioactive gas.
We are in the process of testing Ne and Ar as alternate cooling gases.
In the KSTAR scenarios investigated so far, the effect of Ne on detachment has been hard to observe as small gas puffs do not actuate enough on the \ac{SOL} plasma but if the gas puffing is increased, we suddenly observe disruption due to too much cooling inside the separatrix.
Increased challenges when trying to form radiative divertors or execute detachment control have been noted by other studies \cite{eldon_2025_ppcf}, with possible explanations including differences in penetration and ionization potential compared to other species \cite{casali_2020_pop,casali_2022_nf}.
Some of the issues surrounding use of neon as a radiator have been mitigated by using it in conjunction with another impurity species \cite{Eldon_2023_NME}.

