\section{Discussion}{\label{sec:discussion}

With the rapid advancement in several key areas of plasma physics and fusion devices, the need for advanced control systems for such devices has become more apparent. This requires a more systematic approach to controller development and testing. There is a strong need to initiate such development of controllers and simulate closed-loop tests to understand the performance and constraints before getting access to test them on the machine. For the devices that are still in the design phase, it would be beneficial to involve controller development in the design phase to accommodate control constraints, requirements, and wishes in the design of the device, its sensors, and actuators.

In this work, we have presented the first glimpse of our software suite that can be used for controller development. We have tested this process in simulation with the toy model of density control by utilizing time-dependent SOLPS simulations to study the dynamics of electron density evolution when gas is puffed in the chamber. There are known limitations to the fidelity of SOLPS simulation results though, and validation with experimental data from existing devices is required to make sure that the time response from the simulations matches the physical systems. This comparison will be used to determine if further physics, e.g., running the kinetic neutral particle tracing component of SOLPS in a time-dependent manner\cite{Park2024}, are required to improve the fidelity for unverifiable cases for future devices.

Additionally, there is a limitation to our current method of core profile extrapolation which uses a static profile extension based on gradients at the edge and magnetic equilibrium data. However, core profile transport of gas particles is a more nuanced topic and if core data is important for the diagnostic, it should be done with more sophistication. In particular, the time dynamics of particle diffusion in the core would be different from that in the edge profiles measured by SOLPS, so the data inside the core might have additional delays that should be modeled while extrapolating the data in the time dimension. In the future, we will explore coupling to a 1.5D core transport solver such as FASTRAN\cite{park2018_FASTRAN} or ASTRA\cite{ASTRA} to get a more accurate core profile evolution. This type of coupling has already been done in the CESOL framework\cite{park2018_CESOL} where FASTRAN, EPED, and C2 codes are coupled, and in Park et al.\cite{park2023} where FASTRAN, EPED, and SOLPS codes are coupled.

We have used our model of synthetic diagnostic of the interferometer to read the electron density data from SOLPS simulation like a real-world interferometer would. Our software suite also includes a mode of embedded Langmuir probes for this purpose and many more synthetic diagnostics are in the plans to be added shortly, such as bolometers, electron cyclotron emission spectroscopy, Thomson scattering, etc. Additionally, we have modeled the gas injection system with realistic effects of latency and dribbling due to long pipe lengths which could be very useful in modeling actuators for controller development of large devices like ITER\cite{holtkap_2007_fed}.

We have developed two model-based control strategies that could mitigate the effect of long actuator latency and dynamic non-linearity. \ac{PVLC} is utilizaing the additional model and latency information to linearly fit to the current state vector and predict future output from the plant. This method is close to Kalman\cite{Kalman1960} filtering and Smith predictor\cite{Smith1957} but it deviates from both in subtle manner. \ac{PVLC} is a fully linear and fast algorithm that can run in realtime easily, however, it would not be able to handle (atleast in present form) multiple actuators with different latencies, high dynamic non-linearity in actuators, and capability to adapt model in realtime. However, preset linear time varying models can be used in PVLC if it is well known how the plant model will change throughout a shot.

On the other hand, \ac{MPC} is much more flexible technique that can handle all drawbacks of \ac{PVLC}. We can potentially look into adapting model in realtime based on auxiliary diagnostics and deliberate perturbations. However, \ac{MPC} cost minimization would be slow and results into running MPC in short bursts of feedforward based on intermediate feedback. Both these algorithms will be improved to make them more flexible in use and have high noise and state drift tolerance while providing high bandwidth control even with slow actuators.

This project aims to develop a heat flux controller for future fusion devices like SPARC\cite{Creely_2020}. For this purpose, we would require SOLPS simulations with fuel and impurity gas puffing and measurements of heat flux using various synthetic diagnostics. We would model the heat flux system using advanced system identification techniques such as SINDy (Sparce Identification of Nonlinear Dynamics)\cite{brunton_2016}. Periodic updates in the system model might also be beneficial based on state vector location. We would then use this reduced model to design and test various controllers in closed-loop simulations. This will help us in designing an optimum robust controller without using much of the device test time and minimizing risk of critical failures.

However, the possible scope of this project does not end with a heat flux controller. We have made various generic tools on the IMAS data model to help the future development of synthetic diagnostics and actuators for any device. With such additions, and the use of simulation software or physics models that are compatible with the IMAS data model, controller development for many different problems can be carried out. Our tools can also be used by SOLPS users and diagnostic operation teams to help interpret existing data. We have developed this software package using open source packages, and open source language Julia\cite{bezanson_2017}, to encourage contributions from the community worldwide. We hope that this work will be a stepping stone for more advanced controller development scenarios in the future.