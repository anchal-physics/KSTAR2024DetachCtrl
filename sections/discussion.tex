\section{Discussion}
\label{sec:discussion}

In this paper, we described re-using \Afrac as a reliable control variable for detachment control provided that real-time ion saturation current measurements are available from the Langmuir probes and the strike point is controlled well enough that such an array can be used to estimate peak $I_{sat}$.
It can be seen from panels (a) and (c) in Fig.\ref{fig:sysid_afrac}, Fig.\ref{fig:sysid_sm}, and Fig.\ref{fig:detctrl_sm} that when the total injected impurity amount crosses a rough threshold of about $1\times10^{20}$ particles, the plasma boundary shape is deformed such that strike point on outer divertor starts drifting downwards (and the inner strike point moves upwards) even though the X-point is held in place by the magnetic shape control system.
% This change in divertor leg angle and corresponding changes to the rest of the nearby boundary is sometimes described as rotation of the boundary about the X-point.  % I don't think you need this but IDK.
Thus, \Afrac controller is best suited with an outer strike point control system commissioned on the device, as was demonstrated in the previous carbon divertor \cite{Eldon_2022_PPCF}.
Unfortunately, direct strike point position control (rather than X-point position control), was not yet commissioned for the new divertor at the time of these tests.
It is also important to keep note of the position of strike point and the width of the ion saturation current profile on the divertor.
In our case, we estimated that the ion saturation current profile width is about 16~mm, just enough to ensure that at least one of the Langmuir probes is always inside the wetted area from the \ac{SOL} plasma.
% \textcolor{red}{The strike \emph{point} has no width. The Isat profile has a FWHM of $\approx$16~mm. Other vocabulary that can be used in general situations similar to this are things like profiles at the divertor target plate, Jsat distributions, downstream Jsat, divertor profiles, etc. Comparing the Isat or Jsat profile to the probe spacing puts an upper bound on error in estimating the peak Jsat due to bad alignment between probes and strike point. BTW the pitch angle of the field changes across the plate so the probes don't have the same effective area normal to the magnetic field, even if they have the same physical dimensions. So the Jsat profile could be deformed a little compared to the Isat profile. If the X-point is close to the target, the deformation will be more severe. The probes near the strike point will tend to have lower effective area (although exceptions can be created with weird shapes on flexible devices like d3d and tcv) since the poloidal field is weaker near the X-point. This will probably make the Jsat have a sharper peak than Isat. And Jsat is more intrinsic to the plasma. You don't have to write any of this; just don't think that Isatprofile = Jsatprofile * constant.}
Even then, it can be seen that at around 12.4s in Fig.\ref{fig:sysid_afrac} and at around 10s in Fig.\ref{fig:detctrl_afrac} that as the strike point moves from OD9 (Z=-1.225m) to OD11(Z=-1.2125m), the corresponding \Afrac value shows a sharp decline and then recovery, probably due to the peak passing through the middle of the two probes.
This effect is small enough that our existing controller was able to fix it, but it shows a potential source of error in system identification and might also cause loss of control if the sudden transition can excite an unstable oscillation of the \ac{UGF}.
For future applications of this controller, we are working on including real-time spatial analysis of the $J_{sat}$ profile, potentially informed with profile shapes from high-fidelity simulations from SOLPS-ITER or UEDGE.

In our experimental session, since strike point control had not been commissioned, we attempted real-time empirical profile analysis with strike point sweeps.
But we found that the actuation strength and response time of the poloidal field coils at KSTAR do not allow for large enough and fast enough strike point sweeps.
% \textcolor{red}{Do you want to mention that we attempted real-time empirical profile analysis with strike point sweeps, but found that KSTAR's coils could make big+fast enough sweeps? This would probably work at d3d, but now who cares because it won't work in the long pulse devices. I am in favor of documenting good ideas that don't work so others don't try them without expanding them to solve the issues we found. It also shows we prepared/tried to prepare for a lot of possible gotchas. Things got weird but we really did think a lot of things through in our designs.}

It should be noted that in the application of \Afrac controller method on KSTAR, tuning the overall factor to \Afrac so that it reports 1.0 when fully attached was trickier than the case for full carbon environment KSTAR \cite{Eldon_2022_PPCF}.
We noticed offsets in the outputs of Langmuir probes which changed from shot to shot, and thus ensuring the correct normalizing factor for \Afrac became harder.
This was the reason why we had to change the factor for \Afrac after shot \#35853 as also mentioned in the caption of Fig.\ref{fig:sysid_afrac}.
After this experience, we have now added an online offset estimator and subtraction for all probe signals, which measures the offset before the plasma breakdown and ensures that the zero offset is correct on the probes.
This issue is likely due to electrical connectivity problems with the probes which also showed other issues during the campaign, but still, this experience should be noted for future reproduction and improvements.

Another point of uncertainty in \Afrac model could be the magnetic connection length between the upstream (outboard midplane) and divertor in the 2PM \cite{Leonard_2018_PPCF}.
This length is kept fixed in the model and while we did not change the plasma boundary shape much from the previous test \cite{Eldon_2022_PPCF} in carbon divertor KSTAR, it still makes it a potential source of error in wider usage in future.
The real-time equilibrium calculations during shot provided by RTEFIT is being upgraded to also output this magnetic connection length so that in future the model gets more accurate and time varying information about this important parameter.
% \textcolor{red}{The magnetic connection length between the upstream (outboard midplane) and divertor in the TPM is one of the important inputs, but it would've changed as we tried to optimize the shape. I think this could've been part of the problems. In carbon, I used a shape that had been dialed in pretty well thanks to colleagues like Junghoo Hwang and Matthias Bernert, and the preliminary sys ID shots I took. In tungsten, we were trying to do many changes at once to keep the scenario together. We are doing an upgrade to have RTEFIT output this connection length $L_\parallel$ so that we won't have to recalibrate this aspect of the model during shape changes.}

Although Langmuir probes might not be able to survive future burning plasma experiments, they are still a valuable tool for studying detachment control experiments for ease of installation and operation in experimental devices.
Even in burning plasma devices, sacrificial Langmuir probes can be used in initial device commissioning and preparation of base scenarios at low power, although they are expected to fail shortly after exposure to the intensity of full heat and neutron fluxes.
Knowing what a stable detached scenario should be like could significantly decrease the time required to commission controllers based on other control variables, and give a baseline level of detachment control performance to compare them to.
Thus, Langmuir probe based control might provide a foothold in future device commissioning, as it has been shown to be useful on many devices \cite{Eldon_2021_NME, Guillemaut_2017_PPCF, Yuan_2020_FED, Khodunov_2021_PPCF}.
The good results from \Afrac controller as seen in Fig.\ref{fig:detctrl_afrac} could also motivate further research in similar biased electrode measurement methods of SOL plasma such as biased divertor plates \cite{Toi_2023_NF, Cui_2024_NF} which behave like larger area Langmuir probes and can withstand harsher conditions in comparison to small tip area probes.

We also demonstrated using a machine-learning-based surrogate model, DivControlNN, which infers from a large database of 2D UEDGE simulations for estimating hard to infer quantities in the plasma, such as heat flux on the divertor, for controlling detachment level with real-time feedback.
% \textcolor{red}{KSTAR did have IR thermography for heat flux before; not sure if it works in tungsten but be careful with calling heat flux inaccessible. Might want to rephrase.}
As of the writing of this manuscript, this detachment control method is the first of its kind ever implemented and will act as a stepping stone for future deployments.
This is an important step in the direction of achieving detachment control in future burning plasma reactors which would have very limited means of measuring the detachment level due to space constraints and harsh environment.

We have identified critical weak points in the prototype of DivControlNN and the control infrastructure required to utilize this model, and we are working on improving these aspects for future tests.
One likely mistake we made during the experiments was treating the long dead time reported by DivControlNN heat flux output in response to gas puff (Fig.\ref{fig:sysid_sm}) as an overestimate.
Since DivControlNN had a constant zero impurity fraction concentration, it was solely relying on line integrated core electron density information for responding to changes.
While ion saturation currents provide local divertor information fast, the core electron density response to gas puff would have additional transport timescales and thus DivControlNN output might truly have a larger daed time.
This could have resulted in the high controller gain that saturated the gas response in the test (Fig.\ref{fig:detctrl_sm}).

We are in the process of creating a new 2D UEDGE database of KSTAR with a tungsten divertor and considering multiple charged states of additional impurities such as nitrogen, neon, and argon.
New models would be trained on the expanded database and acquired experimental data from this campaign, with the input of injected gas flow instead of impurity fraction to simplify the use case of these models.
We would also work with the KSTAR team to improve PCS communication infrastructure so that accurate real-time values of injected power are available to our models.

The initial success of the neural network surrogate model in detachment control motivates and corroborates similar studies, simulations, and training of other models for providing fast estimates of plasma parameters, for quick decision-making in the control room during experiments, as well as, for potential use in other control systems where important plasma properties are often not accessible directly.
A neural network based control system approach has already been demonstrated in magnetic shape control \cite{Degrave_2022_Nature}.
For \ac{SOL} plasma predictions, machine learning surrogate models were first pioneered using 1D UEDGE simulations \cite{Zhu_2022_JPP}, serving as the proof-of-principle study that paved the way for DivControlNN presented here, which is based on 2D UEDGE simulations.
More recently, model based on Hermes-3 \cite{Dudson_2024_CPC} simulations of MAST-U \cite{Holt_2024_NF} and neural partial differential equation solver for TCV \cite{Poels_2023_NF} have been reported and are under further development.

Another major focus of future experiments would be to use noble gases in detachment control.
N$_2$, while being excellent at cooling the SOL plasma in conventional tokamaks, would not be a good impurity to seed in tritium fueled plasma due to the formation of tritiated ammonia \cite{Pitts_2019_NME}.
Such tritiated ammonia would require additional tritium reclaiming processes which would reduce the duty cycle of reactors, as well as, pose additional risks in handling a radioactive gas.
% \textcolor{red}{Check whether it's actually more dangerous than any other tritium contamination or just much harder to process to reclaim the precious tritium. And yes, more tritium processing is more opportunity for something to go wrong, so technically it might increase the danger level, but this isn't necessarily by a lot. Is that really significant? I would check to be sure before talking about danger.}
We are in the process of testing Ne and Ar as alternate cooling gases.
In the KSTAR scenarios investigated so far, the effect of Ne on detachment has been hard to observe as small gas puffs do not actuate enough on the \ac{SOL} plasma but if the gas puffing is increased, we suddenly observe disruption due to too much cooling inside the separatrix.
% Commenting out the following for now. Verifying this.
% More widely, there has been recent interest in using pellets for impurity injection for detachment control to reduce the large lag time and response time associated with gas puffing.
% Another alternative is dropping the impurity in the form of solid powder, such as Boron.
% However, the investigation on its use for this purpose is still in a preliminary phase and poses additional challenges in terms of long lag time due to free fall and accumulation of unused powder in the device.
% \textcolor{red}{Who's doing impurity pellets? The disruption mitigation ones don't count.} 
