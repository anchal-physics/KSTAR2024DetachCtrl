\section{Discussion}
\label{sec:discussion}

In this paper, we described re-using \Afrac as a reliable control variable for detachment control provided that realtime ion saturation current measurements are available from the Langmuir probes and the strike point is controlled good enough that such an array can be used to calculate \Afrac. I can be seen from panels (a) and (c) in Fig.\ref{fig:sysid_afrac}, Fig.\ref{fig:sysid_sm}, and Fig.\ref{fig:detctrl_sm} that when the total injected impurity amount crosses a rough threshold of about $1\times10^{20}$ particles, the plasma shape starts to rotate such that strike point on outer divertor starts drifting downwards even though the X-point is held in place by the magnetic shape control system. Thus, \Afrac controller is best suited with a outer strike point control system commissioned on the device which was not the case for the KSTAR esperimental campaign in which we tested these controllers.

Although, langmuir probes might not be able to survive future burning plasma experiments, sacrificial probes can still be used in commissioning controllers based on other control variables in such devices given the success and reliability of this control method being demonstrated in several devices so far\needcite. The good results from \Afrac controller as seen in Fig.\ref{fig:detctrl_afrac} could also motivate further research in similar biased electrode measurement methods of SOL plasma such as biased divertor plates\needcite and biased ring electrodes\needcite.

We were also able to perform a preliminary test on using DivControlNN output as a control variable for detachment control. This is an important step in the direction of achieving detahcment control in future burning plasma reactors which would have very limited means of measuring the detachment level. We have identified critical weak points in the prototype of DivControlNN and the control infrastructure required to utilize this model, and we are working on improving these aspects for future tests. This further motivates focus on training surrogate models on wider set of input parameters and plasma scenarios for possible use in other control systems as well as an alternate way to provide fast diagnostic outputs in the control room during experiments for quick decision making.

Another major focus of future experiments would be to use nobel gases in detachment control as N$_2$ while being optimum in cooling the SOL plasma is not allowed in burning plasma devices due to formation of tritiated ammonia that poses radiactive dangers. We are in the process of testing Ne and Ar as alternate colloing gases. More widely, there has been recent interest in using pellets for impurity injection for detachment control to reduce the large lag time and response time associated with gas puffing. Another alternative is dropping the impurity in the form of solid powder, such as Boron, although the investigation on it's use for this purpose is still in preliminary phase and poses additional challenges in terms of long lag time due to free fall and accumulation of unused powder in the device.