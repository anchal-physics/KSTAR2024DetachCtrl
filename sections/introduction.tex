
\section{Introduction}
\label{sec:introduction}

Burning plasma tokamaks like ITER\cite{holtkap_2007_fed}, SPARC\cite{Creely_2020}, and the various DEMO concepts will generate a high amount of power exhaust coming out in the \ac{SOL} and going towards the divertor. This presents a challenge for divertor design where maximum perpendicular heat load should be limited to 10--15~MW~$\mathrm{m}^{-2}$\cite{pitts_2019_nme}. This requires a majority of the heat flux to be dissipated before it reaches the divertor in a carefully controlled manner. Sufficient dissipation leads to the phenomenon known as divertor detachment which generally involves seeding impurity gases to dissipate the energy and momentum of the exhaust plasma. However, the quantity of the impurity gas needs to be controlled precisely to avoid excessive impurity accumulation in the core plasma\cite{eldon_2023_nme} which, beyond reducing fusion yield by diluting the fuel, can lead to power loss due to radiation and collapse of H-mode and even a disruption which may damage or destroy the tokamak plasma-facing components. 


Several successful attempts at controlling detachment levels in experimental reactors have been achieved with various impurity gases\cite{kallenbach_2012_nf,ravensbergen_2021_nc, eldon_2022_ppcf}. However, the control techniques used so far have mostly relied on linear controllers which are tuned empirically. Such controller development workflow has two limitations. First, the system identification requires experimental data from a previous shot on the same device, under similar conditions. In practice, this has involved operating the device attached mode for some time and documenting the transition into detachment. Such a strategy cannot be used on reactor-class devices where the divertor is expected to be used in detached mode always due to heat load limitations. Secondly, data measured from a single shot of experiment would be insufficient to reliably capture the non-linear dynamics of the system and it would be too expensive to run multiple shots just for system identification purposes. This limits on how complex a controller can be designed based on scarce available data.
% Thus developing a model predictive controller is challenging for such systems. 

In this work, we try to address this issue in controller development workflow by utilizing time-dependent \ac{SOL} plasma simulation software, such as \ac{SOLPS-ITER}\cite{bonnin_2016_pfr}, to study various scenarios, ultimately without empirical data from the device, and developing a model predictive controller based on the simulation results. We would create a non-linear adaptive reduced model of the system based on the simulation that can in-principle foresee system evolution. We can then use such a model in closed-loop simulations with various controller designs to optimally construct and tune a model predictive controller. Such a controller can foresee various non-linearities, delays in actuation, and limitations in actuator strength and directionality. Most importantly, this controller development workflow can be utilized at the design stage of a future reactor, to help with key design choices of actuator and sensor placement, thus involving the control design process before the construction and/or operation of the device. 

In this paper, we describe our first attempt with this workflow on a toy model of electron density control for SPARC using an interferometer as the sensor and gas puffing as the actuator. In section \ref{sec:infrastructure}, we describe a set of software packages we have developed to interface with \ac{SOLPS-ITER} through an \ac{IMAS} data model, implemented synthetic diagnostics on that \ac{IMAS} data model, and our non-linear gas injection model. In section \ref{sec:model}, we describe the identification of the reduced physics model for core electron density, and the model-based control strategies that we have developed for managing actuator latency and non-linearity. We also show closed-loop simulation of the controllers with a non-linear plant model and test it with noise injection and moving target reference. In section \ref{sec:discussion}, we provide the possible improvements in system identification and controller design and discuss some future steps for heat flux controller development.