
\section{Introduction}
\label{sec:introduction}

Burning plasma tokamaks such asITER\cite{Holtkamp_2007_FED}, SPARC\cite{Creely_2020_JPP}, and the various DEMO concepts are estimated to exhaust very high heat flux in the \ac{SOL} towards the divertor target. Significant research has been carried over designing such divertors to withstand the high heat flux and ITER project has decided to use Tungsten as the divertor material. Thus, experimental reactors such KSTAR are in the process of upgrading their divertor and vessel walls to tungsten to design control systems and perform plasma studies in the presence of tunsgten contamination. Even with high melting point of tungsten, it is estimated that perpendicular heat flux on the divertor should be limited to 10--15~MW~$\mathrm{m}^{-2}$\cite{pitts_2019_nme} and the electron temperature at the divertor should remain within 8 eV\cite{Brezinsek_2019_NF} to avoid sputtering of tungsten and subsequent contamination of core plasma.

The heat flux reaching the divertor is typically reduced by puffing in impurity gases to dissipate energy and momemntum of the exhaust plasma in \ac{SOL}. This phenomenon is called divertor detachment. However, excessive impurity seeding could result in core contamination which not only reduces the fusion yield but can also cause cooling of core plasma resulting in collapse of H-mode or disruption of plasma confinement which can damage plasma facing components and vacuum vessel. Thus, it is important to carefully control the amount of impurity injected to keep the divertor cool while not contaminating the core plasma too much.

Divertor detachment is a fairly matured field with several different methods being demonstrated with different sensors and acturators in different experimental reactors. Radiated power control using bolometer measurements was shown in ASDEX Upgrade\cite{Kallenbach_2012_NF}, C-III emission based radiation front control in TCV\cite{Ravensbergen_2021_NC}, divertor electron temperature control using triple-tip Langmuir probes was demonstrated in EAST\cite{Eldon_2021_NME} \textbf{ETC ETC (ADD all latest feedback controlled experiments in other devies  with different control variables here.)} At KSTAR, the detachment control has been achieved using ion saturation current from realtime langmuir probes for calculating attachment fraction, \Afrac \cite{Eldon_2022_PPCF}. This approach has been previously demonstrated in JET\cite{Guillemaut_2017_PPCF}, EAST\cite{Yuan_2020_FED}, and DIII-D\cite{Eldon_2021_NME}. We have extended this approach in KSTAR with tungsten divertor and further tested a new prototpye newral network surrogate model of UEDGE 2D, DivControlNN\cite{Zhu_2025_InPrep}, for realtime heat flux estimation at the outer divertor.

This paper is organized as following. In Sec.\ref{sec:control_variables}, we describe the experimental setup and the definition of different control variables used for detachment control. In Sec.\ref{sec:sysid}, we describe our experimental shots used for identifying the system and using the fitted plant model to tune a PI controller using frequency response for closed-loop stability analysis and optimization. In Sec.\ref{sec:results}, we show the results of our detachment control attempts. And finally, in Sec.\ref{sec:discussion}, we discuss our results, the possible improvements we can make in future, and other interesting contemporary work and ideas in the field of detachment control.