\section{Introduction}
\label{sec:introduction}

Burning plasma tokamaks such as ITER \cite{Holtkamp_2007_FED}, SPARC \cite{Creely_2020_JPP}, and the various DEMO \cite{Federici_2014_FED} and \ac{FPP} \cite{Buttery_2021_NF} concepts are estimated to exhaust very high heat flux in the \ac{SOL} towards the divertor target.
Heat exhausted from the core plasma is rapidly conducted along the open field lines in the \ac{SOL} toward the divertor targets.
To withstand the high heat flux, the divertor target plates are planned to be made out of tungsten, which has a high melting point, good resilience against erosion by the plasma, and relatively low tritium retention compared to other well-studied materials like carbon.
Using ITER's tungsten divertor as an example, it is estimated that for steady-state operation, the constant heat flux reaching the divertor target has to be below 10-15 MW/m$^2$ \cite{Pitts_2019_NME} to avoid surface melting and structural damage to the divertor plates.
Additionally, tungsten being a very high-Z material poses contamination challenges for the core plasma and it is important to develop operation strategies that limit the tungsten sputtering, especially in the divertor region where hot plasma interacts with the tungsten surface in a very narrow region of the order of a few mm \cite{Eich_2013_NF}.
Therefore, the electron temperature at the target plate must be below the tungsten sputtering threshold, which is 8~eV \cite{Brezinsek_2019_NF} for sputtering by deuterium but lower for heavier ions, to minimize contamination of the core plasma.
Thus, experimental reactors such as KSTAR are in the process of changing their divertor and main chamber walls from carbon to tungsten to facilitate study of plasma scenarios, operations, and control in the presence of a reactor-relevant wall.

The heat flux reaching the divertor as well as $T_e$ near the target are both typically reduced by puffing in gas in the \ac{SOL} region to dissipate energy and momentum from the exhaust plasma through ionization, charge exchange, and radiation.
%As the puffed gas neutrals travel toward the core plasma, they radiate energy based on local electron temperature and density.
%Hydrogenic fuel and helium exhaust atoms are ionized at low temperatures in the \ac{SOL} and divertor, and thus offer little radiative cooling except at the edges where the plasma has already cooled.
%Thus for effective and fast cooling, impurity gases such as nitrogen, neon, and argon are puffed which can dissipate heat through radiation farther away from the divertor.
As fuel and helium ash ions are weak radiators, more efficient radiating impurity species such as noble gases will be seeded to increase radiated power and thus decrease heat load conducted to the divertor target.
%In the presence of such radiative dissipation, the heat load conducted by the plasma to the divertor reduces.
Additionally, high density (facilitated both by adding gas and by cooling the plasma at $\approx$constant pressure) dissipates momentum and reduces the total ion flux which impinges on the divertor.
When these dissipation process become significant, recombination occurs and the divertor begins to be shielded from the plasma by a population of neutrals: the primary plasma-neutral interaction zone \emph{detaches} from the solid target plate.
When only part of the surface is detached, the plasma is said to be partially detached, while if the ion flux is almost completely stopped with higher neutral gas pressure, it is said to be fully detached.

It is important though to keep the amount of impurity gases injected into the vessel to a minimum as higher gas injection eventually leads to more impurity reaching in the pedestal region of the plasma.
This can lead to rapid cooling which can collapse H-mode and could also lead to disruption of the plasma confinement.
Such sudden loss of plasma confinement can cause damage to the plasma-facing components.
Thus, it is important to carefully control the amount of impurity injected to keep the divertor cool while not contaminating the core plasma too much.

There are two key approaches to controlling divertor conditions.
The first is to control the sources that determine the fluxes that reach the divertor, such as radiated power throughout the plasma, sometimes divided between the divertor volume, \ac{SOL}, and within the \ac{LCFS}.
This has been successfully demonstrated in various machines:
using the bolometer chords in divertor region in Alcator C-Mod \cite{Goetz_1999_POP}, JT-60U \cite{Asakura_2009_NF}, ASDEX Upgrade \cite{Kallenbach_2012_NF} and DIII-D \cite{Eldon_2019_NME},
using AXUV diodes in EAST \cite{Wu_2018_NF},
using VUV N VII line emission in JET \cite{Maddison_2011_NF}, and
using C-III emission radiation front measured using MANTIS in TCV \cite{Ravensbergen_2021_NC}.

The second way is to directly control key parameters at the divertor target plates.
This has been demonstrated widely in several machines as well:
using divertor plate temperature measurements with surface thermocouples in Alcator C-Mod \cite{Brunner_2017_NF},
using surface electron temperature measurements with triple-tip Langmuir probes in EAST \cite{Eldon_2021_NME} or $T_e$ very close to the surface with divertor Thomson scattering in DIII-D \cite{Eldon_2017_NF},
using ion saturation current measurements from embedded Langmuir probes in JET \cite{Guillemaut_2017_PPCF}, EAST \cite{Yuan_2020_FED}, DIII-D \cite{Eldon_2021_NME}, and COMPASS \cite{Khodunov_2021_PPCF}.
In KSTAR, the ion saturation current measurements along with core electron density, injected power, and local magnetic field were used to calculate a derived control variable, \ac{Afrac}, which was used to control the detachment \cite{Eldon_2022_PPCF}.
This technique has the added benefit that the rollover ion saturation current does not need to be calculated or estimated prior to the shot and thus this technique has the potential for wider applicability in different scenarios and other machines.
In this work, we have re-used this technique in our experiments at KSTAR with a tungsten divertor to test the robustness of this control variable in the presence of high-Z contamination from tungsten.

In this work, we have also tested a new technique that uses a machine-learning-based surrogate model, DivControlNN \cite{zhu_2025_pop}.
This model integrates measurements from several real-time inputs to run through a large database of 2D UEDGE \cite{Rognlien_1999_PP} simulations and provide a real-time estimate of the heat flux reaching the divertor plates along with several other key plasma parameters upstream in \ac{SOL} and at the two divertors.
We tested a prototype of this model with training and input limitations in KSTAR and demonstrated detachment control for the first time using such a surrogate model.
This paves the way for utilizing such models in future reactors that will have a very limited set of sensors available for control systems.

This paper is organized as follows.
In Sec.\ref{sec:control_variables}, we describe the experimental setup and the definition of different control variables used for detachment control.
In Sec.\ref{sec:sysid}, we describe our experimental shots used for identifying the system and using the fitted plant model to tune a PI controller using frequency response for closed-loop stability analysis and optimization.
In Sec.\ref{sec:results}, we show the results of our detachment control attempts.
Finally, in Sec.\ref{sec:discussion}, we discuss our results, the possible control and technical improvements we can make in the future, and how these results can aid in designing further experiments on KSTAR and other tokamaks.