
\section{Introduction}
\label{sec:introduction}

Burning plasma tokamaks such asITER\cite{Holtkamp_2007_FED}, SPARC\cite{Creely_2020_JPP}, and the various DEMO concepts are estimated to exhaust very high heat flux in the \ac{SOL} towards the divertor. Significant research has been carried over designing such divertors to withstand the high heat flux and ITER project has decided to use Tungsten as the divertor material. Thus, experimental reactors such KSTAR are in the process of upgrading their divertor and vessel walls to tungsten to design control systems with tungsten present in the vessel and perform plasma study experiments with tunsgten contamination. Even with high melting point of tungsten, it is estimated that perpendicular heat flux on the divertor should be limited to 10--15~MW~$\mathrm{m}^{-2}$\cite{pitts_2019_nme} and the electron temperature at the divertor should remain within 8 eV\cite{Brezinsek_2019_NF} to avoid sputtering of tungsten and subsequent contamination of core plasma.

The heat flux reaching the divertor is typically reduced by puffing in impurity gases to dissipate energy and momemntum of the plasma in \ac{SOL} region. However, excessive impurity seeding could result in core contamination which not only reduces the fusion yield but can also cause cooling of core plasma resulting in collapse of H-mode or disruption in some cases. Thus, it is important to carefully control the amount of impurity injected to keep the divertor cool while not contaminating the core plasma. This control is generally referred as detachment control as the divertor leg is detached from the divertor target leg and controlled to a specific level of detachment.

Several successful attempts at controlling detachment levels in experimental reactors have been achieved with various impurity gases\cite{Kallenbach_2012_NF,Ravensbergen_2021_NC, Eldon_2022_PPCF}. These experiments were not conducted in tungsten divertor conditions though, where contamination from tungsten can potential cause issues in the control technique. In this work, we demonstrated detachment control in KSTAR 2024 experimental campaign where a tungsten divertor was installed.

This paper is organized as following. In Sec.\ref{sec:control_variables}, we describe the experimental setup and the definition of different control variables used for detachment control. In Sec.\ref{sec:sysid}, we describe our experimental shots used for identifying the system and using the fitted plant model to tune a PI controller using frequency response for closed-loop stability analysis and optimization. In Sec.\ref{sec:results}, we show the results of our detachment control attempts. And finally, in Sec.\ref{sec:discussion}, we discuss our results, the possible improvements we can make in future, and other interesting contemporary work and ideas in the field of detachment control.