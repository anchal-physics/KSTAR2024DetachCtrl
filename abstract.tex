\begin{abstract}

We have demonstrated detachment control in KSTAR with the new Tungsten divertor using different control variables. First variable is the attachment frction calculated from \ac{2PM} using the ion saturation current density from divertor langmuir probes. Second, we used realtime power radiation fraction in the divertor region measured using raw \ac{IRVB} data of particular channels summed together without any use of tomographic inversion. Third, we used a ML surrogate mode trained on 70,000 UEDGE simulations that provided the heatflux at the divertor target as one of the outputs. We also attemped \maybe{demonstrated} detachment control using Neon which is the desirable impurity species for Tritium fueled future reactors such as \ac{ITER}. The basic controller developed by us can be \maybe{has been} used by other plasma experiments to obtain low Tungsten concentration in the plasma due to sputtering at the divertor. We present data on reduction in Tungsten levels with the use of our detachment controller.
\end{abstract}