\begin{abstract}
KSTAR has recently undergone an upgrade to use a new Tungsten divertor to run experiments in ITER-relevant scenarios.
Even with a high melting point of Tungsten, it is important to control the heat flux impinging on tungsten divertor targets to minimize sputtering and contamination of the core plasma.
Heat flux on the divertor is often controlled by increasing the detachment of \ac{SOL} plasma from the target plates.
In this work, we have demonstrated successful detachment control experiments using two different methods.
The first method uses attachment fraction as a control variable which is estimated using ion saturation current measurements from embedded Langmuir probes in the divertor.
The second method uses a novel machine-learning-based surrogate model of 2D UEDGE simulation database, DivControlNN.
We demonstrated running inference operation of DivControlNN in realtime to estimate heat flux at the divertor and use it to feedback impurity gas to control the detachment level.
We present interesting insights from these experiments including a systematic approach to tuning controllers and discuss future improvements in the control infrastructure and control variables for future burning plasma experiments.
\end{abstract}